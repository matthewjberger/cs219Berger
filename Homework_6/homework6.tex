\documentclass[14pt]{article}

\usepackage{fancyhdr}
\usepackage{extramarks}
\usepackage{amsmath}
\usepackage{amsthm}
\usepackage{amsfonts}
\usepackage{amssymb}
\usepackage{tikz}
\usepackage[plain]{algorithm}
\usepackage{algpseudocode}
\usepackage{enumitem}
\usepackage{relsize}
\usepackage{scrextend}
\usepackage{graphicx}

\usetikzlibrary{automata,positioning}

%
% Basic Document Settings
%

\topmargin=-0.45in
\evensidemargin=0in
\oddsidemargin=0in
\textwidth=6.5in
\textheight=9.0in
\headsep=0.25in

\linespread{1.1}

\pagestyle{fancy}
\lhead{\hmwkAuthorName}
\chead{\hmwkClass\ (\hmwkClassInstructor): \hmwkTitle}
\rhead{\firstxmark}
\lfoot{\lastxmark}
\cfoot{\thepage}

\renewcommand\headrulewidth{0.4pt}
\renewcommand\footrulewidth{0.4pt}

\setlength\parindent{0pt}

%
% Create Problem Sections
%

\newcommand{\enterProblemHeader}[1]{
    \nobreak\extramarks{}{Problem \arabic{#1} continued on next page\ldots}\nobreak{}
    \nobreak\extramarks{Problem \arabic{#1} (continued)}{Problem \arabic{#1} continued on next page\ldots}\nobreak{}
}

\newcommand{\exitProblemHeader}[1]{
    \nobreak\extramarks{Problem \arabic{#1} (continued)}{Problem \arabic{#1} continued on next page\ldots}\nobreak{}
    \stepcounter{#1}
    \nobreak\extramarks{Problem \arabic{#1}}{}\nobreak{}
}

\setcounter{secnumdepth}{0}
\newcounter{partCounter}
\newcounter{homeworkProblemCounter}
\setcounter{homeworkProblemCounter}{1}
\nobreak\extramarks{Problem \arabic{homeworkProblemCounter}}{}\nobreak{}

%
% Homework Problem Environment
%
% This environment takes an optional argument. When given, it will adjust the
% problem counter. This is useful for when the problems given for your
% assignment aren't sequential. See the last 3 problems of this template for an
% example.
%
\newenvironment{homeworkProblem}[1][-1]{
    \ifnum#1>0
        \setcounter{homeworkProblemCounter}{#1}
    \fi
    \section{Problem \arabic{homeworkProblemCounter}}
    \setcounter{partCounter}{1}
    \enterProblemHeader{homeworkProblemCounter}
}{
    \exitProblemHeader{homeworkProblemCounter}
}

%
% Homework Details
%   - Title
%   - Due date
%   - Class
%   - Section/Time
%   - Instructor
%   - Author
%


\newcommand{\hmwkClass}{CS 219}
\newcommand{\hmwkTitle}{Homework\ \#5}
\newcommand{\hmwkDueDate}{October 5th, 2016}
\newcommand{\hmwkDueTime}{4:00pm}
\newcommand{\hmwkClassInstructor}{Dr. Egbert}
\newcommand{\hmwkAuthorName}{Matthew J. Berger}

%
% Title Page
%

\title{
    \vspace{2in}
    \textmd{\textbf{\hmwkClass:\ \hmwkTitle}}\\
    \normalsize\vspace{0.1in}\small{Due\ on\ \hmwkDueDate\ at \hmwkDueTime}\\
    \vspace{0.1in}\large{\textit{\hmwkClassInstructor}}
    \vspace{3in}
}

\author{\textbf{\hmwkAuthorName}}
\date{}

\renewcommand{\part}[1]{\textbf{\large Part \Alph{partCounter}}\stepcounter{partCounter}\\}

%
% Various Helper Commands
%

% Useful for algorithms
\newcommand{\alg}[1]{\textsc{\bfseries \footnotesize #1}}

% Alias for the Solution section header
\newcommand{\solution}{\textbf{\large Solution:}}

% Alias for answers
\newcommand{\answer}{\textbf{\large Answer: }}

\begin{document}

\maketitle

\pagebreak

\begin{homeworkProblem}
	\textbf{Review Questions}
	\begin{itemize}
		\item[4.1.)]\begin{enumerate}[label=\alph*)]
			\item When ADD r3, r5, r12 is executed what happens to the value that was in r3?
			\item What is the effect (result) of executing ADD r3, r5, r5?
			\item What is the effect (result) of executing ADD r3, r3, r3?
		\end{enumerate}
		\answer
		\begin{enumerate}[label=\alph*)]
			\item The value that was in r3 becomes the sum of the values stored in r5 and r12.
			\item The value stored in r5 is doubled and stored in r3. The original value of r5 remains unchanged.
			\item The value stored in r3 is doubled in place. r3 now contains the doubled value.
		\end{enumerate}

		\item[4.2.)]\begin{enumerate}[label=\alph*)]
			\item When MOV r11, r2 is executed what happens to the value that was in r11?
			\item What is the effect (result) of executing MOV r4, \#28?
			\item What is the effect (result) of executing MOV r3, r3?
		\end{enumerate}
		\answer
		\begin{enumerate}[label=\alph*)]
			\item The value that was in r11 is lost. The value in r11 is now the same as the value in r2.
			\item The binary pattern equivalent to the decimal number 28 is copied into register r4.
			\item The r3 register is updated with the value it already contains, thus the r3 registered is not changed and contains the same value after this operation.
		\end{enumerate}	
		
		\item[4.3.)] Prepare a program to add together the numbers 127 decimal, 0xe45ad hexadecimal and 2\_10101110010 binary (ARM assembly language uses prefix 2\_ to indicate a number in base 2). Using a development system, assemble your program repeatedly correct it and assemble again until there are no errors. Test that the program behaves correctly. What is the total obtained?
		\answer
		
		\item[4.4.)]\begin{enumerate}[label=\alph*)]
			\item Assume that register r2 holds the value 0x0f45, what is the value in register r5 after the execution of the instruction SUB r5, r2, \#209?
			\item Assume that the register r2 holds the value 0x045, what is the value in register r5 after the execution of the instruction RSB r5, r2, \#209?
			\item What problems arise when register r2 holds 0x0f45 and the instruction RSB r5, r2, \#209 is executed? What will be the value in register r5 after the execution of this instruction?
		\end{enumerate}
		\answer
		\begin{enumerate}[label=\alph*)]
			\item 0x0f45 is equivalent to 3909 in decimal. 3909 - 209 would give the value 3700 which is equivalent to the bit pattern 0000 1110 0111 0100. This is the new value of r5.
			\item 
			\item
		\end{enumerate}
		
		\item[4.5.)]\begin{enumerate}[label=\alph*)]
			\item What is the value in register r8 after the execution of the instruction MVN r8, \#0xf4?
			\item Assume that the register r3 holds the value 0x045 and r10 holds 0xffff. What is the value in register r3 after the execution of the instruction ADD r3, r3, r10?
			\item Describe the form of the value that will be in register r6 after the execution of the instruction ADD r6, r6, \#1?
		\end{enumerate}
		\answer
		\begin{enumerate}[label=\alph*)]
			\item 
			\item 
			\item
		\end{enumerate}		
	\end{itemize}
\end{homeworkProblem}

\end{document}
