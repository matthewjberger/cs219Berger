\documentclass[14pt]{article}

\usepackage{fancyhdr}
\usepackage{extramarks}
\usepackage{amsmath}
\usepackage{amsthm}
\usepackage{amsfonts}
\usepackage{amssymb}
\usepackage{tikz}
\usepackage[plain]{algorithm}
\usepackage{algpseudocode}
\usepackage{enumitem}
\usepackage{relsize}
\usepackage{scrextend}

\usetikzlibrary{automata,positioning}

%
% Basic Document Settings
%

\topmargin=-0.45in
\evensidemargin=0in
\oddsidemargin=0in
\textwidth=6.5in
\textheight=9.0in
\headsep=0.25in

\linespread{1.1}

\pagestyle{fancy}
\lhead{\hmwkAuthorName}
\chead{\hmwkClass\ (\hmwkClassInstructor): \hmwkTitle}
\rhead{\firstxmark}
\lfoot{\lastxmark}
\cfoot{\thepage}

\renewcommand\headrulewidth{0.4pt}
\renewcommand\footrulewidth{0.4pt}

\setlength\parindent{0pt}

%
% Create Problem Sections
%

\newcommand{\enterProblemHeader}[1]{
    \nobreak\extramarks{}{Problem \arabic{#1} continued on next page\ldots}\nobreak{}
    \nobreak\extramarks{Problem \arabic{#1} (continued)}{Problem \arabic{#1} continued on next page\ldots}\nobreak{}
}

\newcommand{\exitProblemHeader}[1]{
    \nobreak\extramarks{Problem \arabic{#1} (continued)}{Problem \arabic{#1} continued on next page\ldots}\nobreak{}
    \stepcounter{#1}
    \nobreak\extramarks{Problem \arabic{#1}}{}\nobreak{}
}

\setcounter{secnumdepth}{0}
\newcounter{partCounter}
\newcounter{homeworkProblemCounter}
\setcounter{homeworkProblemCounter}{1}
\nobreak\extramarks{Problem \arabic{homeworkProblemCounter}}{}\nobreak{}

%
% Homework Problem Environment
%
% This environment takes an optional argument. When given, it will adjust the
% problem counter. This is useful for when the problems given for your
% assignment aren't sequential. See the last 3 problems of this template for an
% example.
%
\newenvironment{homeworkProblem}[1][-1]{
    \ifnum#1>0
        \setcounter{homeworkProblemCounter}{#1}
    \fi
    \section{Problem \arabic{homeworkProblemCounter}}
    \setcounter{partCounter}{1}
    \enterProblemHeader{homeworkProblemCounter}
}{
    \exitProblemHeader{homeworkProblemCounter}
}

%
% Homework Details
%   - Title
%   - Due date
%   - Class
%   - Section/Time
%   - Instructor
%   - Author
%


\newcommand{\hmwkClass}{CS 219}
\newcommand{\hmwkTitle}{Homework\ \#2}
\newcommand{\hmwkDueDate}{September 14, 2016}
\newcommand{\hmwkDueTime}{4:00pm}
\newcommand{\hmwkClassInstructor}{Dr. Egbert}
\newcommand{\hmwkAuthorName}{Matthew J. Berger}

%
% Title Page
%

\title{
    \vspace{2in}
    \textmd{\textbf{\hmwkClass:\ \hmwkTitle}}\\
    \normalsize\vspace{0.1in}\small{Due\ on\ \hmwkDueDate\ at \hmwkDueTime}\\
    \vspace{0.1in}\large{\textit{\hmwkClassInstructor}}
    \vspace{3in}
}

\author{\textbf{\hmwkAuthorName}}
\date{}

\renewcommand{\part}[1]{\textbf{\large Part \Alph{partCounter}}\stepcounter{partCounter}\\}

%
% Various Helper Commands
%

% Useful for algorithms
\newcommand{\alg}[1]{\textsc{\bfseries \footnotesize #1}}

% Alias for the Solution section header
\newcommand{\solution}{\textbf{\large Solution}}

% Alias for answers
\newcommand{\answer}{\textbf{\large Answer: }}

\begin{document}

\maketitle

\pagebreak

\begin{homeworkProblem}
	\large{\underline{Review Questions: }\\\small(All answers are from the slides provided for the course)}
	\begin{itemize}
		\item[1.1)] What, in general terms, is the distinction between computer organization and computer architecture? \\
		\answer \underline{Computer organization} refers to the way a given instruction set is implemented in a particular processor. \underline{Computer Architecture} refers to the attributes of a system visible to the programmer. This is more analagous to an \underline{interface} while the organization describes the \underline{implementation}.
		\item[1.2)] What, in general terms, is the distinction between computer structure and computer function? \\		
		\answer \underline{Computer structure} refers to the way in which componenets relate to each other, while \underline{Computer function} refers to the operation of individual components as part of the structure.
		\item[1.3)] What are the four main functions of a computer? \\
		\answer
		\begin{enumerate}[label=\arabic*)]
			\item \underline{Data Processing}
				\begin{enumerate}[label=\alph*)]
					\item Data may take a wide variety of forms and the range of processing requirements is broad.
				\end{enumerate}
			\item \underline{Data Storage}
				\begin{enumerate}[label=\alph*)]
					\item Short-term
					\item Long-term
				\end{enumerate}
			\item \underline{Data Movement}
				\begin{enumerate}[label=\alph*)]
					\item Input-Output (I/O) - When data is received from or delivered to a device (peripheral) that is directly connected to the computer.
					\item Data Communications - When data is moved
				\end{enumerate}
			\item \underline{Control}
				\begin{enumerate}[label=\alph*)]
					\item A control unit manages the computer's resources and orchestrates the performance of its functional parts in response to instructions.
				\end{enumerate}
			\end{enumerate}
			\pagebreak
			\item[1.4)] List and briefly define the main structural components of a computer. \\
			\answer There are four main structural components of the computer:
				\begin{enumerate}
					\item \underline{CPU} - Controls the operation of the computer and performs its data processing functions.
					\item \underline{Main Memory} - Stores Data
					\item \underline{I/O} - Moves data between the computer and its external environment.
					\item \underline{Sytem Interconnection} - Some mechanism that provides for communication among CPU, main memory, and I/O.

				\end{enumerate}
			\item[1.5)] List and briefly define the main structural components of a processor. \\
			\answer There are four main structural components of a CPU:
			\begin{enumerate}
				\item \underline{Control Unit} - Controls the operation of the CPU and hence the computer.
				\item \underline{Arithmetic Logic Unit} - Performs the computer's data processing function.
				\item \underline{Registers} - Provide storage internal to the CPU.
				\item \underline{CPU Interconnection} - Some mechanism that provides for communication among the control unit, ALU, and registers.
			\end{enumerate}
		
		\item[1.6)] What is a stored program computer? \\
		\answer A stored program computer is one that stores program instructions in memory.
		\item[1.7)] Explain Moore's law. \\
		\answer Moore's law was an observation made in 1965 that the \underline{number of transistors} per square inch on integrated circuits \underline{was doubling each year} since its inception. Gordon Moore (co-founder of Intel) predicted that this would continue to happen ad-infinitum.
		\item[1.8)] List and explain the key characteristics of a computer family. \\
		\answer
			\begin{enumerate}[label=\alph*)]	
			\item Similar or identical instruction set
			\item Similar or identical operating system
			\item Increasing speed
			\item Increasing number of I/O ports
			\item Increasing memory size
			\item Increasing cost
			\end{enumerate}
		\item[1.9)] What is the key distinguishing feature of a microprocessor? \\
		\answer The key distinguishing feature of a microprocessor is that it is a combination of many different circuits integrated into a single chip to provide highly complex functionality.
	\end{itemize}
\end{homeworkProblem}

\pagebreak

\end{document}
