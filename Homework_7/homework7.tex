\documentclass[14pt]{article}

\usepackage{fancyhdr}
\usepackage{extramarks}
\usepackage{amsmath}
\usepackage{amsthm}
\usepackage{amsfonts}
\usepackage{amssymb}
\usepackage{tikz}
\usepackage[plain]{algorithm}
\usepackage{algpseudocode}
\usepackage{enumitem}
\usepackage{relsize}
\usepackage{scrextend}
\usepackage{graphicx}

\usetikzlibrary{automata,positioning}


%
% Basic Document Settings
%

\topmargin=-0.45in
\evensidemargin=0in
\oddsidemargin=0in
\textwidth=6.5in
\textheight=9.0in
\headsep=0.25in

\linespread{1.1}

\pagestyle{fancy}
\lhead{\hmwkAuthorName}
\chead{\hmwkClass\ (\hmwkClassInstructor): \hmwkTitle}
\rhead{\firstxmark}
\lfoot{\lastxmark}
\cfoot{\thepage}

\renewcommand\headrulewidth{0.4pt}
\renewcommand\footrulewidth{0.4pt}

\setlength\parindent{0pt}

%
% Create Problem Sections
%

\newcommand{\enterProblemHeader}[1]{
    \nobreak\extramarks{}{Problem \arabic{#1} continued on next page\ldots}\nobreak{}
    \nobreak\extramarks{Problem \arabic{#1} (continued)}{Problem \arabic{#1} continued on next page\ldots}\nobreak{}
}

\newcommand{\exitProblemHeader}[1]{
    \nobreak\extramarks{Problem \arabic{#1} (continued)}{Problem \arabic{#1} continued on next page\ldots}\nobreak{}
    \stepcounter{#1}
    \nobreak\extramarks{Problem \arabic{#1}}{}\nobreak{}
}

\setcounter{secnumdepth}{0}
\newcounter{partCounter}
\newcounter{homeworkProblemCounter}
\setcounter{homeworkProblemCounter}{1}
\nobreak\extramarks{Problem \arabic{homeworkProblemCounter}}{}\nobreak{}

%
% Homework Problem Environment
%
% This environment takes an optional argument. When given, it will adjust the
% problem counter. This is useful for when the problems given for your
% assignment aren't sequential. See the last 3 problems of this template for an
% example.
%
\newenvironment{homeworkProblem}[1][-1]{
    \ifnum#1>0
        \setcounter{homeworkProblemCounter}{#1}
    \fi
    \section{Problem \arabic{homeworkProblemCounter}}
    \setcounter{partCounter}{1}
    \enterProblemHeader{homeworkProblemCounter}
}{
    \exitProblemHeader{homeworkProblemCounter}
}

%
% Homework Details
%   - Title
%   - Due date
%   - Class
%   - Section/Time
%   - Instructor
%   - Author
%


\newcommand{\hmwkClass}{CS 219}
\newcommand{\hmwkTitle}{Homework\ \#7}
\newcommand{\hmwkDueDate}{October 19th, 2016}
\newcommand{\hmwkDueTime}{4:00pm}
\newcommand{\hmwkClassInstructor}{Dr. Egbert}
\newcommand{\hmwkAuthorName}{Matthew J. Berger}

%
% Title Page
%

\title{
    \vspace{2in}
    \textmd{\textbf{\hmwkClass:\ \hmwkTitle}}\\
    \normalsize\vspace{0.1in}\small{Due\ on\ \hmwkDueDate\ at \hmwkDueTime}\\
    \vspace{0.1in}\large{\textit{\hmwkClassInstructor}}
    \vspace{3in}
}

\author{\textbf{\hmwkAuthorName}}
\date{}

\renewcommand{\part}[1]{\textbf{\large Part \Alph{partCounter}}\stepcounter{partCounter}\\}

%
% Various Helper Commands
%

% Useful for algorithms
\newcommand{\alg}[1]{\textsc{\bfseries \footnotesize #1}}

% Alias for the Solution section header
\newcommand{\solution}{\textbf{\large Solution:}}

% Alias for answers
\newcommand{\answer}{\textbf{\large Answer: }}

\begin{document}

\maketitle

\pagebreak

\begin{homeworkProblem}[12]
	\begin{itemize}
		\item[12.6)] Compare zero-, one-, two-, and three-address machines by writing programs to compute $X=\frac{(A+B \times C)}{(D-E \times F)}$ for each of the four machines. The instructions available for use are as follows:
		\item[]\begin{table}[h]\hfil
			\def\arraystretch{1.5}%
			\begin{tabular}{|l|l|l|l|}
				\hline
				0 Address & 1 Address & 2 Address & 3 Address \\				
				\hline
				PUSH M & LOAD M  & MOVE(X $\leftarrow$ Y)            & MOVE(X $\leftarrow$ Y) \\
				\hline
				POP M  & STORE M & ADD(X $\leftarrow$ X + Y)         & ADD(X $\leftarrow$ Y + Z) \\		
				\hline				
				ADD    & ADD M   & SUB(X $\leftarrow$ X - Y)         & ADD(X $\leftarrow$ Y - Z) \\		
				\hline								
				SUB    & SUB M   & MUL(X $\leftarrow$ X $\times$ Y ) & MUL(X $\leftarrow$ Y $\times$ Z) \\
				\hline
				MUL    & MUL M   & DIV(X $\leftarrow \frac{X}{Y})$   & DIV(X $\leftarrow \frac{Y}{Z}$) \\
				\hline
				DIV    & DIV M   & & \\
				\hline
			\end{tabular}
		\end{table}
		
		\solution
		
		\item[]\begin{table}[h]\hfil
			\def\arraystretch{1.5}%
			\begin{tabular}{|l|l|l|l|}
				\hline
				0 Address & 1 Address & 2 Address & 3 Address \\
				\hline
				PUSH B & LOAD  F & MUL B,C & MUL B,C \\				
				PUSH C & MUL   E & ADD A,B & ADD X,A,B \\
				MUL    & STORE X & MUL E,F & MUL E,F \\
				PUSH A & LOAD  D & SUB D,E & SUB D,E \\
				ADD    & SUB   X & DIV A,D & DIV X,D \\
				PUSH D & STORE X & & \\
				PUSH E & LOAD  C & & \\
				PUSH F & MUL   B & & \\
				MUL    & ADD   A & & \\
				SUB    & DIV   X & & \\
				DIV    & & & \\
				POP  X & & & \\
				
				\hline

			\end{tabular}
		\end{table}
		
		\pagebreak
		
		\item [12.7)] Consider a hypothetical computer with an instruction set of only two n-bit instructions. The first bit specifies the opcode, and the remaining bits specify one of the 2n-1 n-bit words of main memory. The two instructions are as follows:
		\begin{itemize}
			\item \textbf{SUBS X} = Subtract the contents of location X from the accumulator, and store the result in location X and the accumulator.
			\item \textbf{JUMP X} = Place address X in the program counter.		
		\end{itemize}
		A word in main memory may contain either an instruction or a binary number in twos complement notation. Demonstrate that this instruction repertoire is reasonably complete by specifying how the following operations can be programmed. Pick any two from a.) to e.).
		\begin{enumerate}[label=\alph*)]
			\item Data transfer: Location X to accumulator, accumulator to location X.
			\item Addition: Add contents of location X to accumulator.
			\item Conditional branch.
			\item Logical OR.
			\item I/O Operations.
		\end{enumerate}
		
		\solution
		
		\begin{itemize}
			\item[a.)] \underline{\textbf{Data Transfer:}} Location X to accumulator, accumulator to location X. Y,Z are used as scratch space left at zero. X is moved to the accumulator using:
			\begin{enumerate}[label=\arabic*.)]
				\item subs Z
				\item subs Z
				\item subs X
				\item subs Z
				\item subs Z
				\item subs X
			\end{enumerate}
		    and the accumulator to X using:
			\begin{enumerate}[label=\arabic*.)]
				\item subs Z
				\item subs X
				\item subs X
				\item subs Z
				\item subs X
				\item subs Z
				\item subs Z
				\item subs X
			\end{enumerate}
			\pagebreak
			\item[b.)] \underline{\textbf{Addition:}} Add contents of location X to accumulator.
			Addition will be equivalent to the negative accumulator minus X.
			\begin{enumerate}[label=\arabic*.)]
				\item subs Y
				\item subs Z
				\item subs Z
				\item subs X
				\item subs Y
				\item subs Z
				\item subs Z
				\item subs X
				\item subs Z
				\item subs Z
				\item subs Y
			\end{enumerate}
			
		\end{itemize}
		
	\end{itemize}
\end{homeworkProblem}

\begin{homeworkProblem}[13]
	\begin{itemize}
		\item[13.1)] Given the following memory values and a one-address machine with an accumulator, what values do the following instructions load into the accumulator?\\
		
			Word 20 contains 40.\\
			Word 30 contains 50.\\
			Word 40 contains 60.\\
			Word 50 contains 70.
		\begin{enumerate}[label=\alph*.)]
			\item LOAD IMMEDIATE 20
			\item LOAD DIRECT 20
			\item LOAD INDIRECT 20
			\item LOAD IMMEDIATE 30
			\item LOAD DIRECT 30
			\item LOAD INDIRECT 30						
		\end{enumerate}
		
		\solution
		
		\begin{enumerate}[label=\alph*.)]
			\item 20
			\item 40
			\item 60
			\item 30
			\item 50
			\item 70
		\end{enumerate}
		
		\pagebreak
		
		\item[13.3)] An address field in an instruction contains decimal value 14. Where is the corresponding operand located for:
		\begin{itemize}
			\item immediate addressing?
			\item direct addressing?
			\item indirect addressing?
			\item register addressing?
			\item register indirect addressing?		
		\end{itemize}
		
		\solution
		
		\begin{itemize}
			\item The address field
			\item Memory location 14
			\item The memory location whose address is in the memory location 14
			\item Register 14
			\item The memory location whose address is in register 14
		\end{itemize}
		
		\item[13.5)] A PC-relative mode branch instruction is 3-bytes long. The address of the instruction, in decimal, is 256028. Determine the branch target address if the signed displacement in the instruction is -31.
		
		\solution
		
		The next instruction is at \underline{256031} and the branch is \underline{-31}, so the next address is $256031-31=\underline{\textbf{256000}}$.
		
		\item[13.7)] How many times does the processor need to refer to memory when it fetches and executes an indirect-address-mode instruction is:
			\begin{enumerate}[label=\alph*)]
				\item A computation requiring a single operand
				\item A branch
			\end{enumerate}
			
		\solution
			\begin{enumerate}[label=\alph*)]
				\item Fetch instruction, Fetch effective address, Fetch operand. \underline{\textbf{3 memory references}}
				\item Fetch instruction, Fetch effective address and transfer to PC. \underline{\textbf{2 memory references}}
			\end{enumerate}
			
		\item[13.11)] Consider a processor that includes a base with indexing addressing mode. Suppose an instruction is encountered that employs this addressing mode and specifies a displacement of 1970, in decimal. Currently the base and index register contain the decimal numbers 48,022 and 8, respectively. What is the address of the operand?
		
		\solution
		
		Base + index + signed displacement = $48,022 + 8 + 1970 = 50,000_{10} =$ \underline{\textbf{0xC350}}
	\end{itemize}
\end{homeworkProblem}

\end{document}
