\documentclass[14pt]{article}

\usepackage{fancyhdr}
\usepackage{extramarks}
\usepackage{amsmath}
\usepackage{amsthm}
\usepackage{amsfonts}
\usepackage{amssymb}
\usepackage{tikz}
\usepackage[plain]{algorithm}
\usepackage{algpseudocode}
\usepackage{enumitem}
\usepackage{relsize}
\usepackage{scrextend}
\usepackage{graphicx}
\usepackage{listings}

\usetikzlibrary{automata,positioning}

%
% Assembly language code listing
%
\lstset{language=[x86masm]Assembler}

%
% Basic Document Settings
%

\topmargin=-0.45in
\evensidemargin=0in
\oddsidemargin=0in
\textwidth=6.5in
\textheight=9.0in
\headsep=0.25in

\linespread{1.1}

\pagestyle{fancy}
\lhead{\hmwkAuthorName}
\chead{\hmwkClass\ (\hmwkClassInstructor): \hmwkTitle}
\rhead{\firstxmark}
\lfoot{\lastxmark}
\cfoot{\thepage}

\renewcommand\headrulewidth{0.4pt}
\renewcommand\footrulewidth{0.4pt}

\setlength\parindent{0pt}

%
% Create Problem Sections
%

\newcommand{\enterProblemHeader}[1]{
    \nobreak\extramarks{}{Problem \arabic{#1} continued on next page\ldots}\nobreak{}
    \nobreak\extramarks{Problem \arabic{#1} (continued)}{Problem \arabic{#1} continued on next page\ldots}\nobreak{}
}

\newcommand{\exitProblemHeader}[1]{
    \nobreak\extramarks{Problem \arabic{#1} (continued)}{Problem \arabic{#1} continued on next page\ldots}\nobreak{}
    \stepcounter{#1}
    \nobreak\extramarks{Problem \arabic{#1}}{}\nobreak{}
}

\setcounter{secnumdepth}{0}
\newcounter{partCounter}
\newcounter{homeworkProblemCounter}
\setcounter{homeworkProblemCounter}{1}
\nobreak\extramarks{Problem \arabic{homeworkProblemCounter}}{}\nobreak{}

%
% Homework Problem Environment
%
% This environment takes an optional argument. When given, it will adjust the
% problem counter. This is useful for when the problems given for your
% assignment aren't sequential. See the last 3 problems of this template for an
% example.
%
\newenvironment{homeworkProblem}[1][-1]{
    \ifnum#1>0
        \setcounter{homeworkProblemCounter}{#1}
    \fi
    \section{Problem \arabic{homeworkProblemCounter}}
    \setcounter{partCounter}{1}
    \enterProblemHeader{homeworkProblemCounter}
}{
    \exitProblemHeader{homeworkProblemCounter}
}

%
% Homework Details
%   - Title
%   - Due date
%   - Class
%   - Section/Time
%   - Instructor
%   - Author
%


\newcommand{\hmwkClass}{CS 219}
\newcommand{\hmwkTitle}{Homework\ \#10}
\newcommand{\hmwkDueDate}{November 23rd, 2016}
\newcommand{\hmwkDueTime}{4:00pm}
\newcommand{\hmwkClassInstructor}{Dr. Egbert}
\newcommand{\hmwkAuthorName}{Matthew J. Berger}

%
% Title Page
%

\title{
    \vspace{2in}
    \textmd{\textbf{\hmwkClass:\ \hmwkTitle}}\\
    \normalsize\vspace{0.1in}\small{Due\ on\ \hmwkDueDate\ at \hmwkDueTime}\\
    \vspace{0.1in}\large{\textit{\hmwkClassInstructor}}
    \vspace{3in}
}

\author{\textbf{\hmwkAuthorName}}
\date{}

\renewcommand{\part}[1]{\textbf{\large Part \Alph{partCounter}}\stepcounter{partCounter}\\}

%
% Various Helper Commands
%

% Useful for algorithms
\newcommand{\alg}[1]{\textsc{\bfseries \footnotesize #1}}

% Alias for the Solution section header
\newcommand{\solution}{\textbf{\large Solution:}}

% Alias for answers
\newcommand{\answer}{\textbf{\large Answer: }}

\begin{document}

\maketitle

\pagebreak

\begin{homeworkProblem}

1.) Write a program called SUB64 to subtract the 64-bit integer in memory locations 0x0150 and 0x154 from the 64-bit integer in
0x0160 and 0x0164. Store the result in memory location 0x0170 and 0x0174.\\

\solution

\begin{table}[h]
\begin{tabular}{|c|}
\hline
\begin{lstlisting}
; Find8 
; D. Egbert ver 1.1 11/14/2016
org 100h
section .text
; beginning address of code = 0x0100
;*******************************************************
; put your code here
start:
		MOV	BX,[LSTL]
		MOV	CL,0			;USE CL FOR LARGEST VALUE
LP1:	MOV	AL,[BX+DLST]
		CMP	CL,AL			;TEST IF BL > AL
		JA	LPC				;CONTINUE IF NOT
		MOV	CL,AL			;ELSE STORE NEW MAX
LPC:	DEC	BX
		JGE	LP1
		MOV	[MAXV],CL
; end of your code
;*******************************************************		
ILP:	JMP	ILP				;infinite loop
;
TIMES 50H -($-$$) DB 0
section .data
; beginning address of data = 0x0150
;*******************************************************
; put your data items here
LSTL:	DB	14
MAXV:	DB	0
DLST:	DB	254,5,25,250,100,150,30,200,253,15,23,46,73,175,0
; end of your data
;*******************************************************
\end{lstlisting} \\
\hline
\end{tabular}
\end{table}
\end{homeworkProblem}

\pagebreak

\begin{homeworkProblem}

2.) Write a program called COMBINE that combines the low-order nibbles of the four bytes in memory locations 0x0150 to 0x0153
into a single 16-bit word. The nibbles should be ordered low-to-high in the result beginning with the data from location 0x0150. Store
the result as 16-bits in memory location 0x0154. \\

\solution

\begin{table}[h]
\begin{tabular}{|c|}
\hline
\begin{lstlisting}
; Find8 
; D. Egbert ver 1.1 11/14/2016
org 100h
section .text
; beginning address of code = 0x0100
;*******************************************************
; put your code here
start:
		MOV	BX,[LSTL]
		MOV	CL,0			;USE CL FOR LARGEST VALUE
LP1:	MOV	AL,[BX+DLST]
		CMP	CL,AL			;TEST IF BL > AL
		JA	LPC				;CONTINUE IF NOT
		MOV	CL,AL			;ELSE STORE NEW MAX
LPC:	DEC	BX
		JGE	LP1
		MOV	[MAXV],CL
; end of your code
;*******************************************************		
ILP:	JMP	ILP				;infinite loop
;
TIMES 50H -($-$$) DB 0
section .data
; beginning address of data = 0x0150
;*******************************************************
; put your data items here
LSTL:	DB	14
MAXV:	DB	0
DLST:	DB	254,5,25,250,100,150,30,200,253,15,23,46,73,175,0
; end of your data
;*******************************************************
\end{lstlisting} \\
\hline
\end{tabular}
\end{table}
\end{homeworkProblem}

\pagebreak

\begin{homeworkProblem}

3.) Write a program called FIND to find the larger of two signed bytes. Assume the two bytes are in memory locations 0x0150 and
0x0151. Store the larger of the two in memory location 0x0152.\\

\solution

\begin{table}[h]
\begin{tabular}{|c|}
\hline
\begin{lstlisting}
; Find8 
; D. Egbert ver 1.1 11/14/2016
org 100h
section .text
; beginning address of code = 0x0100
;*******************************************************
; put your code here
start:
		MOV	BX,[LSTL]
		MOV	CL,0			;USE CL FOR LARGEST VALUE
LP1:	MOV	AL,[BX+DLST]
		CMP	CL,AL			;TEST IF BL > AL
		JA	LPC				;CONTINUE IF NOT
		MOV	CL,AL			;ELSE STORE NEW MAX
LPC:	DEC	BX
		JGE	LP1
		MOV	[MAXV],CL
; end of your code
;*******************************************************		
ILP:	JMP	ILP				;infinite loop
;
TIMES 50H -($-$$) DB 0
section .data
; beginning address of data = 0x0150
;*******************************************************
; put your data items here
LSTL:	DB	14
MAXV:	DB	0
DLST:	DB	254,5,25,250,100,150,30,200,253,15,23,46,73,175,0
; end of your data
;*******************************************************
\end{lstlisting} \\
\hline
\end{tabular}
\end{table}
\end{homeworkProblem}

\pagebreak

\begin{homeworkProblem}

4.) Write a program called LSHIFT to shift logically the 32-bit contents of memory location 0x0150 left according to the 8-bit shift
count stored in memory location 0x0154 and store the results at memory address 0x0158. \\

\solution

\begin{table}[h]
\begin{tabular}{|c|}
\hline
\begin{lstlisting}
; Find8 
; D. Egbert ver 1.1 11/14/2016
org 100h
section .text
; beginning address of code = 0x0100
;*******************************************************
; put your code here
start:
		MOV	BX,[LSTL]
		MOV	CL,0			;USE CL FOR LARGEST VALUE
LP1:	MOV	AL,[BX+DLST]
		CMP	CL,AL			;TEST IF BL > AL
		JA	LPC				;CONTINUE IF NOT
		MOV	CL,AL			;ELSE STORE NEW MAX
LPC:	DEC	BX
		JGE	LP1
		MOV	[MAXV],CL
; end of your code
;*******************************************************		
ILP:	JMP	ILP				;infinite loop
;
TIMES 50H -($-$$) DB 0
section .data
; beginning address of data = 0x0150
;*******************************************************
; put your data items here
LSTL:	DB	14
MAXV:	DB	0
DLST:	DB	254,5,25,250,100,150,30,200,253,15,23,46,73,175,0
; end of your data
;*******************************************************
\end{lstlisting} \\
\hline
\end{tabular}
\end{table}
\end{homeworkProblem}

\pagebreak

\begin{homeworkProblem}

5.) Write a program called FIND8 to find the largest unsigned 8-bit word in a list. The list begins at address 0x0154. The length of the
list is stored in an 8-bit variable at address 0x0150. Store the largest entry in memory location 0x0152. \\

\solution

\begin{table}[h]
\begin{tabular}{|c|}
\hline
\begin{lstlisting}
; Find8 
; D. Egbert ver 1.1 11/14/2016
org 100h
section .text
; beginning address of code = 0x0100
;*******************************************************
; put your code here
start:
		MOV	BX,[LSTL]
		MOV	CL,0			;USE CL FOR LARGEST VALUE
LP1:	MOV	AL,[BX+DLST]
		CMP	CL,AL			;TEST IF BL > AL
		JA	LPC				;CONTINUE IF NOT
		MOV	CL,AL			;ELSE STORE NEW MAX
LPC:	DEC	BX
		JGE	LP1
		MOV	[MAXV],CL
; end of your code
;*******************************************************		
ILP:	JMP	ILP				;infinite loop
;
TIMES 50H -($-$$) DB 0
section .data
; beginning address of data = 0x0150
;*******************************************************
; put your data items here
LSTL:	DB	14
MAXV:	DB	0
DLST:	DB	254,5,25,250,100,150,30,200,253,15,23,46,73,175,0
; end of your data
;*******************************************************
\end{lstlisting} \\
\hline
\end{tabular}
\end{table}
\end{homeworkProblem}

\pagebreak

\begin{homeworkProblem}

6.) Write a program called FIND32 to find the largest unsigned 32-bit word in a list. The list begins at address 0x0160. The length of
the list is stored in an 8-bit variable at address 0x0150. Store the largest entry in memory location 0x0154. \\

\solution

\begin{table}[h]
\begin{tabular}{|c|}
\hline
\begin{lstlisting}
; Find8 
; D. Egbert ver 1.1 11/14/2016
org 100h
section .text
; beginning address of code = 0x0100
;*******************************************************
; put your code here
start:
		MOV	BX,[LSTL]
		MOV	CL,0			;USE CL FOR LARGEST VALUE
LP1:	MOV	AL,[BX+DLST]
		CMP	CL,AL			;TEST IF BL > AL
		JA	LPC				;CONTINUE IF NOT
		MOV	CL,AL			;ELSE STORE NEW MAX
LPC:	DEC	BX
		JGE	LP1
		MOV	[MAXV],CL
; end of your code
;*******************************************************		
ILP:	JMP	ILP				;infinite loop
;
TIMES 50H -($-$$) DB 0
section .data
; beginning address of data = 0x0150
;*******************************************************
; put your data items here
LSTL:	DB	14
MAXV:	DB	0
DLST:	DB	254,5,25,250,100,150,30,200,253,15,23,46,73,175,0
; end of your data
;*******************************************************
\end{lstlisting} \\
\hline
\end{tabular}
\end{table}
\end{homeworkProblem}

\pagebreak

\begin{homeworkProblem}

7.) Write a program called SCAN to scan a list of unsigned bytes and find the smallest and largest entries in the list. The length of the
list is stored in a 16-bit variable at addresses 0x0152 and 0x0154. The list begins at address 0x0160. Store the smallest byte at address
0x0150 and the largest byte at address 0x0151. \\

\solution

\begin{table}[h]
\begin{tabular}{|c|}
\hline
\begin{lstlisting}
; Find8 
; D. Egbert ver 1.1 11/14/2016
org 100h
section .text
; beginning address of code = 0x0100
;*******************************************************
; put your code here
start:
		MOV	BX,[LSTL]
		MOV	CL,0			;USE CL FOR LARGEST VALUE
LP1:	MOV	AL,[BX+DLST]
		CMP	CL,AL			;TEST IF BL > AL
		JA	LPC				;CONTINUE IF NOT
		MOV	CL,AL			;ELSE STORE NEW MAX
LPC:	DEC	BX
		JGE	LP1
		MOV	[MAXV],CL
; end of your code
;*******************************************************		
ILP:	JMP	ILP				;infinite loop
;
TIMES 50H -($-$$) DB 0
section .data
; beginning address of data = 0x0150
;*******************************************************
; put your data items here
LSTL:	DB	14
MAXV:	DB	0
DLST:	DB	254,5,25,250,100,150,30,200,253,15,23,46,73,175,0
; end of your data
;*******************************************************
\end{lstlisting} \\
\hline
\end{tabular}
\end{table}
\end{homeworkProblem}

\pagebreak

\begin{homeworkProblem}

8.) Write a program called COUNT to count the number of characters in a null-terminated ASCII string that are equal to a KEY. The
KEY is stored in memory location 0x0150. The string is stored in memory beginning at address 0x0160. Store the 8-bit count in
memory location 0x0154. (Assume the maximum count is 255.) \\

\solution

\begin{table}[h]
\begin{tabular}{|c|}
\hline
\begin{lstlisting}
; Find8 
; D. Egbert ver 1.1 11/14/2016
org 100h
section .text
; beginning address of code = 0x0100
;*******************************************************
; put your code here
start:
		MOV	BX,[LSTL]
		MOV	CL,0			;USE CL FOR LARGEST VALUE
LP1:	MOV	AL,[BX+DLST]
		CMP	CL,AL			;TEST IF BL > AL
		JA	LPC				;CONTINUE IF NOT
		MOV	CL,AL			;ELSE STORE NEW MAX
LPC:	DEC	BX
		JGE	LP1
		MOV	[MAXV],CL
; end of your code
;*******************************************************		
ILP:	JMP	ILP				;infinite loop
;
TIMES 50H -($-$$) DB 0
section .data
; beginning address of data = 0x0150
;*******************************************************
; put your data items here
LSTL:	DB	14
MAXV:	DB	0
DLST:	DB	254,5,25,250,100,150,30,200,253,15,23,46,73,175,0
; end of your data
;*******************************************************
\end{lstlisting} \\
\hline
\end{tabular}
\end{table}
\end{homeworkProblem}

\pagebreak

\begin{homeworkProblem}

9.) Write a program called ONES to determine the number of bits equal to one in a 32-bit variable. The 32-bit variable is in memory
location 0x0154. Store the 8-bit counter in memory location 0x0150. \\

\solution

\begin{table}[h]
\begin{tabular}{|c|}
\hline
\begin{lstlisting}
; Find8 
; D. Egbert ver 1.1 11/14/2016
org 100h
section .text
; beginning address of code = 0x0100
;*******************************************************
; put your code here
start:
		MOV	BX,[LSTL]
		MOV	CL,0			;USE CL FOR LARGEST VALUE
LP1:	MOV	AL,[BX+DLST]
		CMP	CL,AL			;TEST IF BL > AL
		JA	LPC				;CONTINUE IF NOT
		MOV	CL,AL			;ELSE STORE NEW MAX
LPC:	DEC	BX
		JGE	LP1
		MOV	[MAXV],CL
; end of your code
;*******************************************************		
ILP:	JMP	ILP				;infinite loop
;
TIMES 50H -($-$$) DB 0
section .data
; beginning address of data = 0x0150
;*******************************************************
; put your data items here
LSTL:	DB	14
MAXV:	DB	0
DLST:	DB	254,5,25,250,100,150,30,200,253,15,23,46,73,175,0
; end of your data
;*******************************************************
\end{lstlisting} \\
\hline
\end{tabular}
\end{table}
\end{homeworkProblem}

\pagebreak

\begin{homeworkProblem}

10.) Write a subroutine called STRLEN that determines the length of a null-terminated ASCII string. Pass the 16-bit start address of the
string to the subroutine in register BX. Return the length, excluding the null byte, in register CX. All registers (except CX) should
return to the calling program unchanged. \\

\solution

\begin{table}[h]
\begin{tabular}{|c|}
\hline
\begin{lstlisting}
; Find8 
; D. Egbert ver 1.1 11/14/2016
org 100h
section .text
; beginning address of code = 0x0100
;*******************************************************
; put your code here
start:
		MOV	BX,[LSTL]
		MOV	CL,0			;USE CL FOR LARGEST VALUE
LP1:	MOV	AL,[BX+DLST]
		CMP	CL,AL			;TEST IF BL > AL
		JA	LPC				;CONTINUE IF NOT
		MOV	CL,AL			;ELSE STORE NEW MAX
LPC:	DEC	BX
		JGE	LP1
		MOV	[MAXV],CL
; end of your code
;*******************************************************		
ILP:	JMP	ILP				;infinite loop
;
TIMES 50H -($-$$) DB 0
section .data
; beginning address of data = 0x0150
;*******************************************************
; put your data items here
LSTL:	DB	14
MAXV:	DB	0
DLST:	DB	254,5,25,250,100,150,30,200,253,15,23,46,73,175,0
; end of your data
;*******************************************************
\end{lstlisting} \\
\hline
\end{tabular}
\end{table}
\end{homeworkProblem}

\pagebreak

\begin{homeworkProblem}

11.) Write a subroutine called REPLACE that processes a null-terminated string of decimal characters and replaces leading zeros with
spaces. Pass the 32-bit address of the string to the subroutine in register BX.  \\

\solution

\begin{table}[h]
\begin{tabular}{|c|}
\hline
\begin{lstlisting}
; Find8 
; D. Egbert ver 1.1 11/14/2016
org 100h
section .text
; beginning address of code = 0x0100
;*******************************************************
; put your code here
start:
		MOV	BX,[LSTL]
		MOV	CL,0			;USE CL FOR LARGEST VALUE
LP1:	MOV	AL,[BX+DLST]
		CMP	CL,AL			;TEST IF BL > AL
		JA	LPC				;CONTINUE IF NOT
		MOV	CL,AL			;ELSE STORE NEW MAX
LPC:	DEC	BX
		JGE	LP1
		MOV	[MAXV],CL
; end of your code
;*******************************************************		
ILP:	JMP	ILP				;infinite loop
;
TIMES 50H -($-$$) DB 0
section .data
; beginning address of data = 0x0150
;*******************************************************
; put your data items here
LSTL:	DB	14
MAXV:	DB	0
DLST:	DB	254,5,25,250,100,150,30,200,253,15,23,46,73,175,0
; end of your data
;*******************************************************
\end{lstlisting} \\
\hline
\end{tabular}
\end{table}
\end{homeworkProblem}

\pagebreak

\begin{homeworkProblem}

12.) Write a program called UNPACK to convert the 16-bit BCD variable in memory locations 0x0150 and 0x0151 to four ASCII
characters with the high-order digit first, beginning in memory location 0x0154. \\

\solution

\begin{table}[h]
\begin{tabular}{|c|}
\hline
\begin{lstlisting}
; Find8 
; D. Egbert ver 1.1 11/14/2016
org 100h
section .text
; beginning address of code = 0x0100
;*******************************************************
; put your code here
start:
		MOV	BX,[LSTL]
		MOV	CL,0			;USE CL FOR LARGEST VALUE
LP1:	MOV	AL,[BX+DLST]
		CMP	CL,AL			;TEST IF BL > AL
		JA	LPC				;CONTINUE IF NOT
		MOV	CL,AL			;ELSE STORE NEW MAX
LPC:	DEC	BX
		JGE	LP1
		MOV	[MAXV],CL
; end of your code
;*******************************************************		
ILP:	JMP	ILP				;infinite loop
;
TIMES 50H -($-$$) DB 0
section .data
; beginning address of data = 0x0150
;*******************************************************
; put your data items here
LSTL:	DB	14
MAXV:	DB	0
DLST:	DB	254,5,25,250,100,150,30,200,253,15,23,46,73,175,0
; end of your data
;*******************************************************
\end{lstlisting} \\
\hline
\end{tabular}
\end{table}
\end{homeworkProblem}

\end{document}
